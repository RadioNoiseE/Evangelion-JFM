%! TeX Program = LuaLaTeX

\documentclass[
    paper = a4paper,
    fontsize = 12Q,
    jafontsize = 12Q,
    %line_length = ?zw,
    %number_of_lines = ?,
    article
]{jlreq}

\usepackage[match]{luatexja-fontspec}
\setmainfont{Linux Libertine O}
\setmainjfont{Source Han Serif SC}[Language = Chinese Simplified, YokoFeatures = {JFM = {eva/smpl, nstd}}]
\setsansfont{Linux Biolinum O}
\setsansjfont{Source Han Sans SC}[Language = Chinese Simplified, YokoFeatures = {JFM = {eva/smpl, nstd}}]
\setmonofont{Courier New}
\setmonojfont{Source Han Sans SC}

\usepackage{listings}
\lstloadlanguages{TeX}
\lstset{
    language = TeX,
    basicstyle = \ttfamily\small,
    breaklines = true,
    numbers = left,
    numberstyle = \tiny,
    stepnumber = 1,
    numbersep = 4pt
}

\usepackage{hyperref}
\hypersetup{
    hidelinks,
    pdftitle = {Evangelion-JFM},
    pdfauthor = {RadioNoiseE},
    pdfsubject = {TeX},
    pdfkeywords = {Japanese Font Metric},
    pdfstartview = FitV
}

\def\空{\quad}
\def\段{\par}
\def\LuaTeX{Lua\kern-.2ex\TeX}

\title{\sffamily Evangelion Japanese Font Metric for \LuaTeX}
\author{\large \url{https://github.com/RadioNoiseE/Evangelion-JFM}}
\date{西历2023年}

\begin{document}
\parindent2\zw

\maketitle

\begin{abstract}
    本文档将介绍名为Evangelion Japanese Font Metric(下简称为``\lstinline|Eva-JFM|'')的JFM文件。其适用于简体中文(以下简称为「简中」)、繁体中文(以下简称为「繁中」)及日文字体。旨在提供一个充分利用\LuaTeX{}-ja的\lstinline|priority|特性,基于标准\cite{jlreq}的同时,同时支持一些罕用特性的JFM文件。文档使用中文及西文撰写。\段
    This document introduces a JFM file named ``Evangelion-JFM'' (hereinafter referred to as ``\lstinline{Eva-JFM}''). It supports Simplified Chinese (hereinafter referred to as ``SC''), Traditional Chinese (hereinafter referred to as ``TC''), and Japanese (hereinafter referred to ad ``JP''). It aims to provide a JFM file that takes full advantage of the \lstinline|priority| and other powerful features from LuaTeX-ja, supporting the well-accepted standards\cite{jlreq}, and supporting some rarely-used features. This documentation is written in both Chinese Simplified and English.
\end{abstract}

\section{背景及略介:Background Knowledge and a Rough Introduction}
→2月9日撰

\begin{thebibliography}{99}
    \addcontentsline{toc}{section}{\refname}
    \bibitem{jlreq} W3C Japanese Layout Task Force~(ed). \newblock Requirements for Japanese Text Layout (W3C Working Group Note), 2011, 2012. \newblock \url{http://www.w3.org/TR/jlreq/}.
\end{thebibliography}

\end{document}
