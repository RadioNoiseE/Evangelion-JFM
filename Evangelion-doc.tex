%! TeX Program = LuaLaTeX

\documentclass{ltjsarticle}
%\documentclass[
%    paper = a4paper,
%    fontsize = 12Q,
%    jafontsize = 12Q,
%    %line_length = ?zw,
%    %number_of_lines = ?,
%    article
%]{jlreq}

\usepackage[match]{luatexja-fontspec}
\setmainfont{Linux Libertine O}
\setmainjfont{Source Han Serif SC}[Language = Chinese Simplified, YokoFeatures = {JFM = eva/{smpl, nstd}}]
\setsansfont{Linux Biolinum O}
%\setsansjfont{Source Han Sans SC}[Language = Chinese Simplified, YokoFeatures = {JFM = eva/{smpl, nstd}}]
\setmonofont[Scale=MatchLowercase, FakeStretch=1.137121, RawFeature=-notdef]{Iosevka Slab}
%\setmonojfont{Source Han Sans SC}

\usepackage{luatexja-adjust}
\ltjenableadjust[priority = true]

\usepackage{listings}
%\lstloadlanguages{TeX}
\lstset{
    %language = TeX,
    basicstyle = \ttfamily\small,
    breaklines = true,
    columns = flexible,
    numbers = left,
    numberstyle = \tiny,
    stepnumber = 1,
    numbersep = 4pt,
    escapechar = §
}

\usepackage{hyperref}
\hypersetup{
    hidelinks,
    pdftitle = {Evangelion-JFM},
    pdfauthor = {RadioNoiseE},
    pdfsubject = {TeX},
    pdfkeywords = {Japanese Font Metric},
    pdfstartview = FitV
}

\long\def\feature#1#2#3{{\smallskip\vbox{\parindent=\zw\hangindent=2\zw\texttt{#1}\quad\texttt{\itshape #2\/}\\\indent#3\smallskip}}}
\def\meta#1{{\normalfont\rmfamily\itshape$\langle$#1\/$\rangle$}}
\def\空{\quad}
\def\段{\par}
\def\LuaTeX{Lua\kern-.2ex\TeX}
\def\pTeX{p\kern-.2ex\TeX}
\def\pdfTeX{pdf\TeX}

\title{\sffamily\bfseries Evangelion Japanese Font Metric for \LuaTeX}
\author{\large \url{https://github.com/RadioNoiseE/Evangelion-JFM}}
\date{西历2023年}

\begin{document}
\parindent=2\zw\parskip=2pt

\maketitle

\begin{abstract}
    本文档将介绍名为Evangelion Japanese Font Metric(下简称为``\textsf{Eva-JFM}'')的JFM文件。其适用于简体中文(以下简称为「简中」)、繁体中文(以下简称为「繁中」)及日文字体。旨在提供一个充分利用\LuaTeX{}-ja的\texttt{priority}特性,基于标准\cite{jlreq}的同时,同时支持一些罕用特性的JFM文件。文档使用中文及西文撰写。\段
    This document introduces a JFM file named ``Evangelion-JFM'' (hereinafter referred to as ``\textsf{Eva-JFM}''). It supports Simplified Chinese (hereinafter referred to as ``SC''), Traditional Chinese (hereinafter referred to as ``TC''), and Japanese (hereinafter referred to ad ``JP''). It aims to provide a JFM file that takes full advantage of the \texttt{priority} and other powerful features from \LuaTeX{}-ja, supporting the widely accepted standards (i.e. \cite{jlreq}), and supporting some rarely-used features. The documentation is written in both Chinese Simplified and English.
\end{abstract}

\section{背景及略介:Background Knowledge and a Rough Introduction}
\TeX{}是高德纳教授于29世纪末开发的强大排版引擎,能够完全满足西文排版的需求。然因时代局限性\footnote{如没有事实上的统一字符编码等。}以及客观原因\footnote{如中日字符集较大,以及书写方式的不同(纵书、横书),标点等。}支持十分有限。为达成中日排版需求,在宏扩展(如\textsf{CJK}等)之外出现了引擎扩展。影响力较大的是\pTeX{}系列。\段
\pTeX{}系列采用虚拟字体的理念,使用\texttt{TFM/VF}映射TrueType或OpenType字体完成排版。其不支持宏配置字体,也不支持直接生成PDF格式文件。但可以满足日本的传统横纵排版需求(工业标准)。\段
\pdfTeX{}则是当时另一个\TeX{}的引擎扩展,支持不经DVI格式直接输出PDF格式的文件。然对Unicode(字符编码)及TrueType、OpenType(「现代」矢量字体格式)的支持繁琐或有限。\LuaTeX{}便是基于\pdfTeX{}的引擎扩展,在原生支持Unicode下提供Lua语言扩展(使能够使用\textsf{fontloader}等模块)支持现代字体。宏配置字体特性由\textsf{luaotfload}宏集提供。它也支持直接生成PDF文件。\段
\LuaTeX{}-ja可看作是对两者的合并。这是一个由日本开发者北川宏典首倡的\LuaTeX{}下的日文支持项目,即将\pTeX{}(大部分)移植到\LuaTeX{}下。由于\LuaTeX{}支持宏配置字体,故不需要\texttt{VF}文件为字体提供映射,但为标点挤压等需求保留并扩展\footnote{如优先挤压(\texttt{priority})特性,及一些特殊字符(如\texttt{parbdd}、\texttt{glue})等。}了JFM文件。\段
本项目就是一个JFM文件。使用Lua的\texttt{callback},将简中、繁中、日文及行间标点、压缩字体特性集中于\texttt{jfm-eva.lua}单个文件中。用户可按需调用特性来完成高质量的中日排版。

\section{安裝及本地配置:Installation and Local Configs}
本项目将源文件托管于GitHub平台,暂未上传至Comprehensive \TeX{} Archive Net(CTAN)。用户可使用
\begin{lstlisting}
    mkdir Evangelion-JFM [&&] cd Evangelion-JFM
    git clone https://github.com/RadioNoiseE/Evangelion-JFM
\end{lstlisting}
获取源文件,再将其放置在本地的\texttt{TEXMF}路径中,如
\begin{lstlisting}
    ~/Library/texlive/2023/texmf-dist/tex/luatex/eva-jfm
\end{lstlisting}
等。最后运行
\begin{lstlisting}
    mktexlsr
\end{lstlisting}
更新本地\TeX{}的\texttt{Ls-R}文件即可。\段
本文件一般情况下无需用户进行本地配置,但若有特殊需求可见\ref{}←☂。

\section{使用:Using}
以下是在\LaTeX{}下使用繁中字体进行直排的示例
\begin{lstlisting}
    \usepackage{luatexja-fontspec, luatexja-adjust}
    \setmainjfont{Source Han Serif TC}[Language = Chinese Traditional, TateFeatures = {JFM = eva/{vert, trad, nstd}}]
    \ltjenableadjust[priority = true]
\end{lstlisting}
(注意需要调用支持直书的文档类或使用\texttt{\string\tate}命令)。\LuaTeX-ja的JFM语法为:
\begin{lstlisting}
    jfm = §\meta{JFM name}§/{§\meta{JFM features}§}
\end{lstlisting}
而一般情况使用\texttt{\string\setmainjfont}时则为:
\begin{lstlisting}
    \setmainjfont{§\meta{font name}§}[Language = §\meta{language name}§, §\meta{dir}§ = {JFM = §\meta{JFM name}§/{§\meta{JFM features}§}}]
\end{lstlisting}
其中,\meta{font name}自然为需要的字体名称。\meta{language name}在使用日文字体时可忽略,而使用简中、繁中字体时为必填,因\LuaTeX-ja会默认将其覆盖为\texttt{Japanese}选项,而这会带来灾难性的后果\footnote{比如错误的标点位置:日文为冒号及分号中置、其余偏靠,简中是全部偏靠,而繁中则是统统中置。}。\meta{dir}选填\texttt{TateFeatures}(直书)或\texttt{YokoFeatures}(横书)。其后的\meta{JFM name}为调用JFM的文件名\footnote{\LuaTeX-ja会依\texttt{jfm-\meta{JFM name}.lua}的格式来查找该文件。}。最后的\meta{JFM features}选项为选择使用的JFM特性,详细请看第\ref{sec:feat}章。\段
其他情况下设置JFM及其特性请看\LuaTeX-ja文档\cite{luatexja-doc}。

\section{支持特性:Supported Features}\label{sec:feat}
本章节将介绍\textsf{Eva-JFM}的所有特性,分别为:语言特性、方向特性、扩展特性及私有特性。

\subsection{語言特性:Language Features}
本区特性必填且只可填一个。不然则会报错。\段
\feature{jp}{JaPanese}{
    日本语特性。当使用日文字体时需调用该特性。其与简中、繁中区别在于问号「?\inhibitglue」及感叹号「!\inhibitglue」后插入的伸缩胶量。影响特性\texttt{lgp},且对内部分组有影响。
}
\feature{trad}{TRADitional chinese}{
    繁中特性。当使用繁体中文字体时需调用。与简中、日本语特性的区别源于中置的标点。故,对于全部标点左右插入的伸缩胶的量都与简中、日本语不同。针对句点紧挨闭括号、标点位于句末时等皆有优化。
}
\feature{smpl}{SMPLified}{
    简中特性,使用简体中文字体排版时调用。与日本语、繁中特性区别源于分号「;\inhibitglue」及冒号「:\inhibitglue」等全部偏靠从而影响其左右插入伸缩胶的量。\textsf{Eva-JFM}对一些(不该出现的)神奇情况(如两个句号同时出现、开括号后出现问号等)进行优化(?\inhibitglue)。对问号「?\inhibitglue」、感叹号「!\inhibitglue」作了特殊处理。
}

\subsection{方向特性:(Writing) Direction Features}
本分区特性与全部其他特性兼容,可同时调用。\段
\feature{vert}{VERTical writing}{
    直书特性。对标点挤压、分组有影响。直书时必须调用。
}

\subsection{扩展特性:Extended Features}
本区特性依赖\texttt{vert}特性,需同时调用。否则报错。\段
\feature{extd}{EXTenDed font}{
    压缩字体特性。目前仅支持横比纵为100比80的字体压缩\footnote{日本新闻字体,如每日新闻明朝体。}。需同\texttt{extend}(\textsf{luaotfload})或\texttt{FakeStretch}(\textsf{fontspec})同时使用。
}
\feature{lgp}{LineGap Punctuations}{
    行间标点特性。该特性将部分标点「悬挂」至行间。日文字体时与繁、简中字体时会有区别。详见第\ref{sec:lgp}章。
}

\subsection{私有特性:Dark Features}
使用本区特性前请先确保你清楚地知道你在做什么。\段
\feature{nstd}{Non STandarD}{
    忽略标准特性。字体排印标准\cite{jlreq}认为逗号「,\inhibitglue」的压缩权重应比句号「。\inhibitglue」要低。本特性将句号的压缩优先级与逗号交换,使逗号被优先压缩\footnote{考虑逗号、句号在文字系统中占的重量,以及「开明」压缩风格。}。仅在使用\textsf{luatexja-adjust}宏集时有用。
}

\section{行间标点特性:More About Linegap Punctuations}\label{sec:lgp}
本章节将提供更多详细的关于行间标点特性的信息,以及可能出现的问题及其解决方案。

\subsection{关于「悬挂」:About ``Hanging''}
行间标点可见于古籍之中,是将标点符号与直书结合妥协的产物。
传统上悬挂句号「。\inhibitglue」与逗号「,\inhibitglue」。而\textsf{Eva-JFM}特性在繁中、简中特性下会悬挂句号、逗号、顿号「、\inhibitglue」、冒号「:\inhibitglue」及分号「;\inhibitglue」,日文字体下则不悬挂冒号及分号。原因在于日本习惯上将冒号与分号看作「中点类」,直书时横置处理。
本JFM将全部标点悬挂于字体右下位置。详见下一节。

\subsection{悬挂的位置:Hanging Position}

\begin{thebibliography}{99}
    \addcontentsline{toc}{section}{\refname}
    \bibitem{jlreq} W3C Japanese Layout Task Force~(ed). \newblock Requirements for Japanese Text Layout (W3C Working Group Note), 2011, 2012. \newblock \url{http://www.w3.org/TR/jlreq/}.
\end{thebibliography}

\end{document}
